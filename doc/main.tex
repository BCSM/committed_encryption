\documentclass{article}
\usepackage{amsfonts}
\usepackage[operators]{cryptocode}
\usepackage[utf8]{inputenc}

\title{Introduction of the committed encryption}
\author{Weikeng Chen\\
\vspace{0.12em}
for CS294-144 Blockchain course project}
\date{May 2018}

\begin{document}

\maketitle

\section{Motivation}

We want to encrypt one message to several individual parties. With public-key encryption under a group $\mathbb{G}$ and the group generator $g\in\mathbb{G}$, we assume there are three parties, and their public/private key pairs are $(g^{x_1},x_1)$,  $(g^{x_2},x_2)$, and $(g^{x_3},x_3)$, respectively, we can encrypt a message $m$ as follows:
\[
\begin{array}{l}
r_1\sample[0,...,|\mathbb{G}|-1],~r_2\sample[0,...,|\mathbb{G}|-1],~r_3\sample[0,...,|\mathbb{G}|-1],\\
k\sample\{0,1\}^\lambda,\\
c_1=g^{r_1}\parallel\mathsf{KDF}(g^{x_1r_1})\oplus k,\\
c_2=g^{r_2}\parallel\mathsf{KDF}(g^{x_2r_2})\oplus k,\\
c_3=g^{r_3}\parallel\mathsf{KDF}(g^{x_3r_3})\oplus k,\\
c_4=\mathsf{AuthEnc}.\mathsf{Enc}_k(\mathsf{msg}).
\end{array}
\]
where $\mathsf{KDF}(\cdot)$ is a key derivation function and $\mathsf{AuthEnc}$ refers to an authenticated encryption scheme.

For honest encryption, all three parties can decrypt the message using their secret key. However, if the encryption is not honest, different parties may see different messages, we formalize as follows:
\[
\begin{array}{l}
r_1\sample[0,...,|\mathbb{G}|-1],~r_2\sample[0,...,|\mathbb{G}|-1],~r_3\sample[0,...,|\mathbb{G}|-1],\\
k\sample\{0,1\}^\lambda,\\
c_1=g^{r_1}\parallel\mathsf{KDF}(g^{x_1r_1})\oplus k,\\
c_2=g^{r_2}\parallel\mathsf{KDF}(g^{x_2r_2})\oplus k~\boxed{\oplus\Delta_2},\\
c_3=g^{r_3}\parallel\mathsf{KDF}(g^{x_3r_3})\oplus k~\boxed{\oplus\Delta_3},\\
c_4=\mathsf{AuthEnc}.\mathsf{Enc}_k(\mathsf{msg}).
\end{array}
\]
in which the party 2 obtains $\mathsf{k}\oplus\Delta_2$ and the party 3 obtains $\mathsf{k}\oplus\Delta_3$ instead of $\mathsf{k}$. Note that the party 2's key $\mathsf{k}\oplus\Delta_2$ may decrypt $c_4$ into non-$\perp$ result, as the authenticity of authenticated encryption does not necessarily protect against malicious encryption.

We want to provide a multi-recipient encryption that satisfies the following properties: (1) semantic security; (2) party 1 can detect whether $c_2$ and $c_3$ is generated honestly. Therefore, a malicious encryption cam be detected by the party 1 without knowing the secret keys of the party 2 and the party 3.

\section{Our construction, informally}
We encrypt the message as follows with a pseudorandom function:
\[
\begin{array}{l}
s\sample\{0,1\}^\lambda,~k=\mathsf{PRF}_s(1),\\
r_1\sample[0,...,|\mathbb{G}|-1],\\
\qquad c_1=g^{r_1}\parallel\mathsf{KDF}(g^{x_1r_1})\oplus s,\\
r_2=\mathsf{PRF}_s(2)\mod{|\mathbb{G}|},\\
\qquad c_2=g^{r_2}\parallel\mathsf{KDF}(g^{x_2r_2})\oplus k,\\
r_3=\mathsf{PRF}_s(3)\mod{|\mathbb{G}|},\\
\qquad c_3=g^{r_3}\parallel\mathsf{KDF}(g^{x_3r_3})\oplus k,\\
c_4=\mathsf{AuthEnc}.\mathsf{Enc}_k(\mathsf{msg}).
\end{array}
\]
assumed the Hash Diffie-Hellman (HDH) assumption holds on the group $\mathbb{G}$. 

\smallskip
\noindent\textbf{Semantic security.} We can prove the semantic security of this construction with the random oracle heuristic. 

By hybrid arguments, we have the following proof. We assume the distinguisher has all the public keys, but does not have any private keys of the three parties.

\smallskip
\noindent $\mathbf{H}_0$: the honest generation.

\smallskip
\noindent $\mathbf{H}_1$: in this hybrid, we change the $s$ in $c_1$ into $0^\lambda$:
\[
\begin{array}{l}
s\sample\{0,1\}^\lambda,~k=\mathsf{PRF}_s(1),\\
r_1\sample[0,...,|\mathbb{G}|-1],\\
\qquad \boxed{c_1=g^{r_1}\parallel\mathsf{KDF}(g^{x_1r_1}),}\\
r_2=\mathsf{PRF}_s(2)\mod{|\mathbb{G}|},\\
\qquad c_2=g^{r_2}\parallel\mathsf{KDF}(g^{x_2r_2})\oplus k,\\
r_3=\mathsf{PRF}_s(3)\mod{|\mathbb{G}|},\\
\qquad c_3=g^{r_3}\parallel\mathsf{KDF}(g^{x_3r_3})\oplus k,\\
c_4=\mathsf{AuthEnc}.\mathsf{Enc}_k(\mathsf{msg}).
\end{array}
\]

This hybrid is computationally indistinguishable with $\mathbf{H}_0$ because $\mathsf{KDF}(g^{x_1r_1})$ is computationally indistinguishable to a random $\lambda$-bit string assuming HDH. 

\smallskip
\noindent $\mathbf{H}_2$: in this hybrid, we change $k$, $r_2$, and $r_3$ to randomly sampled values and discard $s$:

\[
\begin{array}{l}
\boxed{k\sample\{0,1\}^\lambda,}\\
r_1\sample[0,...,|\mathbb{G}|-1],\\
\qquad c_1=g^{r_1}\parallel\mathsf{KDF}(g^{x_1r_1}),\\
\boxed{r_2\sample[0,...,|\mathbb{G}|-1],}\\
\qquad c_2=g^{r_2}\parallel\mathsf{KDF}(g^{x_2r_2})\oplus k,\\
\boxed{r_3\sample[0,...,|\mathbb{G}|-1],}\\
\qquad c_3=g^{r_3}\parallel\mathsf{KDF}(g^{x_3r_3})\oplus k,\\
c_4=\mathsf{AuthEnc}.\mathsf{Enc}_k(\mathsf{msg}).
\end{array}
\]

This hybrid is computationally indistinguishable with $\mathbf{H}_1$ because of $\mathsf{PRF}$ with hidden seed $s$. 

\smallskip
\noindent $\mathbf{H}_3$: in this hybrid, we hide the key $k$:

\[
\begin{array}{l}
r_1\sample[0,...,|\mathbb{G}|-1],\\
\qquad c_1=g^{r_1}\parallel\mathsf{KDF}(g^{x_1r_1}),\\
r_2\sample[0,...,|\mathbb{G}|-1],\\
\qquad \boxed{c_2=g^{r_2}\parallel\mathsf{KDF}(g^{x_2r_2}),}\\
r_3\sample[0,...,|\mathbb{G}|-1],\\
\qquad \boxed{c_3=g^{r_3}\parallel\mathsf{KDF}(g^{x_3r_3}),}\\
k\sample\{0,1\}^\lambda,\\
c_4=\mathsf{AuthEnc}.\mathsf{Enc}_k(\mathsf{msg}).
\end{array}
\]

This hybrid is computationally indistinguishable with $\mathbf{H}_2$ due to the same reason for $\mathbf{H}_1\stackrel{\mathsf{c}}{\approx}\mathbf{H}_0$.

\smallskip
\noindent $\mathbf{H}_4$: in this hybrid, we change the message into $1^l$, where $l$ is the message size:

\[
\begin{array}{l}
r_1\sample[0,...,|\mathbb{G}|-1],\\
\qquad c_1=g^{r_1}\parallel\mathsf{KDF}(g^{x_1r_1}),\\
r_2\sample[0,...,|\mathbb{G}|-1],\\
\qquad c_2=g^{r_2}\parallel\mathsf{KDF}(g^{x_2r_2}),\\
r_3\sample[0,...,|\mathbb{G}|-1],\\
\qquad c_3=g^{r_3}\parallel\mathsf{KDF}(g^{x_3r_3}),\\
k\sample\{0,1\}^\lambda,\\
c_4=\mathsf{AuthEnc}.\mathsf{Enc}_k(\boxed{1^l}).
\end{array}
\]

This hybrid is computationally indistinguishable with $\mathbf{H}_3$ due to the message indistinguishability of the authenticated encryption.

According to the hybrid arguments, $\mathbf{H}_0\stackrel{\mathsf{c}}{\approx}\mathbf{H}_4$, but $\mathbf{H}_4$ can be simulated by a PPT simulator without the message. This shows that our construction achieves semantic security under HDH assumption.

\smallskip
\noindent\textbf{Accountability (in our definition).} As mentioned in the motivation section, we want the party 1 to be able to detect malicious encryption, in which different parties will see different results.

We first introduce the problem. In a nutshell, we want to see if $\Delta_1$, $\Delta_2$, ..., and $\Delta_4$ equal to zero pads and whether $k=k'$ in the following ciphertext:

\[
\begin{array}{l}
c_1=g^{r_1}\parallel\mathsf{KDF}(g^{x_1r_1})\oplus s,\\
c_2=g^{r_2}~\boxed{\oplus\Delta_1}\parallel\mathsf{KDF}(g^{x_2r_2})\oplus k ~\boxed{\oplus\Delta_2,}\\
c_3=g^{r_3}~\boxed{\oplus\Delta_3}\parallel\mathsf{KDF}(g^{x_3r_3})\oplus k ~\boxed{\oplus\Delta_4,}\\
c_4=\mathsf{AuthEnc}.\mathsf{Enc}_{k'}(\mathsf{msg}).
\end{array}
\]
where $k=\mathsf{PRF}_s(1)$, $r_2=\mathsf{PRF}_s(2)$, and $r_3=\mathsf{PRF}_s(3)$.

We now introduce the algorithm to detect the malicious encryption.  
\[
\begin{array}{l}
\mathsf{Detect}(c_1,c_2,c_3,c_4,\mathbb{G},g,g^{x_1},g^{x_2},g^{x_3},x_1)
\end{array}
\]
which runs as follows:
\begin{enumerate}
    \item Computes $g^{x_1r_1}$ from $g^{r_1}$ and $x_1$.
    
    \item Recovers $s$ by XOR the second part of $c_1$ with $\mathsf{KDF}(g^{x_1r_1})$.
    
    \item Computes the following pseudorandom values:
    \begin{itemize}
        \item $k=\mathsf{PRF}_s(1)$.
        \item $r_2=\mathsf{PRF}_s(2)$.
        \item $r_3=\mathsf{PRF}_s(3)$.
    \end{itemize}
    
    \item Checks whether the first part of $c_2$ equals to $g^{r_2}$ and the first part of $c_3$ equals to $g^{r_3}$. If not, outputs $\mathsf{CHEAT}$ and terminates. Otherwise, we know $\Delta_1$ and $\Delta_3$ are both zero pads, and the ciphertext has the following format:
    \[
    \begin{array}{l}
    c_1=g^{r_1}\parallel\mathsf{KDF}(g^{x_1r_1})\oplus s,\\
    c_2=\boxed{g^{r_2}}\parallel\mathsf{KDF}(g^{x_2r_2})\oplus k \oplus\Delta_2,\\
    c_3=\boxed{g^{r_3}}\parallel\mathsf{KDF}(g^{x_3r_3})\oplus k \oplus\Delta_4,\\
    c_4=\mathsf{AuthEnc}.\mathsf{Enc}_{k'}(\mathsf{msg}).
    \end{array}
    \]
    
    \item Computes $g^{x_2r_2}$ from $r_2$ and $g^{x_2}$ and $g^{x_3r_3}$ from $r_3$ and $g^{x_3}$. Note that we only use the public keys of the party 2 and the party 3.
    
    \item Recovers $\Delta_2$ by XOR the second part of $c_2$ with $\mathsf{KDF}(g^{x_2r_2})\oplus k$ and $\Delta_4$ by XOR the second part of $c_4$ with $\mathsf{KDF}(g^{x_3r_3})\oplus k$. 
    
    \item Checks if $\Delta_2$ and $\Delta_4$ are both zero pads. If not, outputs $\mathsf{CHEAT}$ and terminates. Otherwise, we know the ciphertext has the following format:
    \[
    \begin{array}{l}
    c_1=g^{r_1}\parallel\mathsf{KDF}(g^{x_1r_1})\oplus s,\\
    c_2=g^{r_2}\parallel\boxed{\mathsf{KDF}(g^{x_2r_2})\oplus k,}\\
    c_3=g^{r_3}\parallel\boxed{\mathsf{KDF}(g^{x_3r_3})\oplus k,}\\
    c_4=\mathsf{AuthEnc}.\mathsf{Enc}_{k'}(\mathsf{msg}).
    \end{array}
    \]
    
    \item Uses $k$ to decrypt $c_4$. If the decrypted result is $\perp$, the ciphertext is not valid, the algorithm outputs $\mathsf{CHEAT}$ and terminates. Otherwise, outputs $\mathsf{ACCEPT}$, as the ciphertext now has the following format the same as the one from honest encryption:
    
    \[
    \begin{array}{l}
    c_1=g^{r_1}\parallel\mathsf{KDF}(g^{x_1r_1})\oplus s,\\
    c_2=g^{r_2}\parallel\mathsf{KDF}(g^{x_2r_2})\oplus k,\\
    c_3=g^{r_3}\parallel\mathsf{KDF}(g^{x_3r_3})\oplus k,\\
    c_4=\mathsf{AuthEnc}.\boxed{\mathsf{Enc}_{k}}(\mathsf{msg}).
    \end{array}
    \]
\end{enumerate}
According to the discussion above, if the detection algorithm outputs $\mathsf{ACCEPT}$, three parties will see the same decrypted result. 
    
\end{document}

